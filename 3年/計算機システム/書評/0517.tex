\documentclass[titlepage]{jarticle}
\usepackage[dvipdfmx]{graphicx}
\usepackage{here}

\title{計算機システム 書評}
\author{41番 鷲尾 優作}
\date{令和3年5月17日}
\begin{document}
\maketitle

\section{図書情報}
一般社団法人インターネットユーザー協会 編\\
保護者のためのあたらしいインターネットの教科書\\

(株)中央経済社\\
2012年5月1日初版発行\\
ISBN978-4-502-69620-6

\section{選択理由}
この本は、対象を保護者にユーザーを子供に絞って書かれている。
初めから大きなバイアスがかかっているゆえに、作者のリテラシー感や
読者をどういった方向に誘導しようとしているのかが他の技術書と異なり
読み応えのある書評となると考えたからである。

\section{総合評価}
\begin{itemize}
    \item リテラシー教育において十分使用できると考える
    \item 非常にかみ砕かれたネットワークの説明が載っている
    \item 恐怖を煽りがち
    \item 内容が実体験を伴っていない想像であるように感じる
\end{itemize}

\section{観点1 メディアリテラシ}
わかりやすい文章でネットワークとは何か、SNSとは何かなど
教科書のような文章である。法律にかかわる説明も多く、インターネット上
での自らの行動がどんな刑事罰を招来する可能性があるかという部分までかなり
事細かに書かれている。

特に、インターネットの匿名性にかかわる部分ではTwitterとはいかなるものなのか
何故匿名なのか、匿名だからといってどんなことをしてはいけないのか
といった重要事項が綺麗に整理されている。

ただし、「~です。~だからです。」という説明を多用しており
2文目の追加説明に個人的な感情が含まれすぎているとも感じた。
情報のフィルタリングの部分は本書のかなり重要な部分であり、
ホワイトリストやブラックリストの解説なども含まれる。
この部分に関しては、「フィルタする側」の意見が色濃く出ておりフィルタする意義は
語られているがフィルタすることによってどんな面倒な影響出るかなどについては
触れられていない。保護者向けの本なので仕方ないことではあるが少し偏りがあると感じた。

\section{観点2 技術説明}
本書の中では、ネットとは何か、フィルタリング、アクセス権、メール、
SNS、生放送について技術の解説がある。\\
もちろん後の説明をわかりやすくするため書かれているものなので技術書解説としては
十分な量ではないが、インターネットを知らない保護者に概要を伝えるには十分であると思う。

\section{全体を通して}
非常に良い本である。\\
下手に保護者向けに絞り込むのではなく、子供に自分で読んでもらう対象としても大丈夫なくらい親切な説明がなされている。
この筆者はなにもインターネットを使うのをやめさせようとしているわけではなく安全に使う方法を紹介したいだけ
だと思うのだが、読んでみた結果としてかなり恐怖をあおる内容に偏っていると感じた。
そして、一般的な事実を平然と述べていることが多く、白々しい文章も多い。

ミクシィなどでてくるので、2021年に読むには少し情報が古いかもしれない。

「迷惑メール防止法」など私自身初めて知るような法律に関する説明が各所に挟まれているが
この部分は端的で感情を感じさせない、非常に良い文章だった。4つほど知らなかった法律に触れることができた。

\section{授業内容に関する意見・要望および感想}
インターネットのに関わる年表を授業中に示していただけるのは楽しい。また我々が触れていない時代のソフトウェアの解説など
小噺が飽きずに授業を受けられている理由と思う。

教科書の付属のノートが教科書の内容と微妙にずれていて使いにくい。

今回の書評課題は最近活字に触れていなかったのでいい機会だった。
他人の書評が読めると書評のうまい書き方が分かりそうな気がするので
ぜひ交換読書をやってみたい。

\end{document}




