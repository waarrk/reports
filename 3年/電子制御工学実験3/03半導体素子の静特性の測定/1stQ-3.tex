\documentclass[titlepage]{jarticle}
\usepackage{h31ec-exp}
\usepackage[dvipdfmx]{graphicx}
\usepackage[yen]{okuverb}
\usepackage{here}

\title{半導体素子の静特性の測定}
\grade{3年41番}%
\author{鷲尾 優作}
\team{}
\date{令和3年6月14日,6月21日,6月28日}
\expdate{令和3年6月28日}
\coauthor{
  39番 & 宮崎 来\\
  42番 & 渡辺 あかり\\
  %34番 & 西脇 光
}

\begin{document}
\maketitle

%目次の出力
\tableofcontents
\newpage

\section{実験A ダイオードとツェナーダイオードの静特性}
\subsection{目的}
\begin{itemize}
    \item pn接合半導体素子(ダイオード)の順方向,逆方向の電圧-電流特性を測定する.
    \item 逆方向の電圧を増加させた時に,ある電圧になると電流が急増する降伏特性を利用したツェナダイオードの電圧-電流特性を測定する.
    \item 交流入力に対する半波整流特性を観測することにより,ダイオードの整流特性について学習する.
\end{itemize}
\subsection{使用器具}
\begin{enumerate}
    \item ダイオード静特性評価回路 EC-01
    \item 直流安定化電源装置 TAKASAGO GPO25-5 帳1 番号195 分類B21
    \item 交流電源装置 KENWOOD AG-203 EC-02
    \item デジタルマルチメータ SANWA CD770 EC-34
    \item オシロスコープ GWINSTEK GDS-1022 No.6
\end{enumerate}
\subsection{ダイオードの静特性}
\subsubsection{順方向の特性}
\subsubsection{逆方向の特性}
\subsection{ツェナーダイオードの静特性}
\subsubsection{順方向の特性}
\subsubsection{逆方向の特性}
\subsection{ダイオードの半波整流特性}
\subsection{考察}

\newpage
\section{実験B トランジスタの静特性}
\subsection{目的}
\begin{itemize}
    \item npn形トランジスタの静特性を測定し,その特性を把握する
    \item エミッタの設置回路について理解する
\end{itemize}
\subsection{使用器具}
\begin{enumerate}
    \item トランジスタ静特性評価回路 EC-07
    \item 直流安定化電源装置 KIKUSUI PMC35-2 L 48-10-1-42
    \item 直流安定化電源装置 TAKASAGO GPO25-5 帳1 番号195 分類B21
    \item デジタルマルチメータ SANWA CD770 EC-26, EC-35
    \item オシロスコープ GWINSTEK GDS-1022 No.2
\end{enumerate}
\subsection{評価回路の構造}
\subsection{$V_{BE}$-$I_B$特性の測定}
\subsection{$V_{CE}$-$I_C$特性の測定}
\subsection{考察}

\newpage
\section{感想}

\begin{thebibliography}{99}
    \bibitem{}皆川正寛,「半導体素子の静特性の測定,令和3年度電子制御工学実験・3年前期テキスト」
    %\bibitem{竹部}竹部啓輔, 「{\TeX}によるレポート作成,令和3年度電子制御工学実験・3年前期テキスト pp.A-1-A-18, (2021/4.
\end{thebibliography}

\end{document}