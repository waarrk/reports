\documentclass[titlepage]{jarticle}
\usepackage{r03ec-exp}
\usepackage[yen]{okuverb}

%
%%% 表紙の記載事項設定
%
% 実験題目  ※レポートを書くときは,まず,タイトルを正しいものに変えましょう
%
\title{{\TeX}によるレポート作成}
% 学年・番号
\grade{3年41番}%
% 氏名
\author{鷲尾 優作}
% 班(後期は班に分かれて実験をする.そのときは,ここに班番号を記入する.)
\team{}
% 提出日
\date{2021年5月11日}
% 実験日
\expdate{2021年4月12日,4月19日,4月26日,5月10日}
% 共同実験者
% グループに分かれて実験をするテーマでは,グループメンバーの番号名前を書く.
\coauthor{}
%
%記載例:
%\coauthor{%
%  2番 & 新潟 花子\\
%  11番 & 三条 次郎}
%%

\begin{document}
\maketitle

\section{はじめに}

本実験では,組版ソフトウェアの{\TeX}を用いて,学生実験などのレポートを作成する方法を学ぶ.本実験を進めるにあたり,レポートを作成するうえで注意すべきことを理解しておく必要がある.

以下,{\TeX}の機能を順に練習していく.

\section{基本練習}

節,小節,小小節には自動的に番号が振られる.

\subsection{段落の分け方}
本文の段落を変えるには空行を入れればよい.

日本語文書の場合,段落の最初の1字分が自動的に字下げされる.\\
強制改行をすると,この行のように字下げが行われない.したがって,強制改行を段落の区切りに使ってはいけない.

\subsection{文字サイズの変更}
\label{sec:文字サイズ}
{\TeX}では標準で10種類の文字サイズが使える.たとえば,{\footnotesize 脚注の文字サイズ(footnotesize)}{\small ちょっと小さい字(small)}{\large ちょっと大きい字(large)}のようにサイズを変えて表示できる.

\subsection{記号}
いくつかの文字はそのままでは出力できないので,特別な書き方をする.\#\$\%\&\_\~{}\{\}

``引用記号''には,ダブルクォーテーションは使わず,``バッククォート(Shift+@)''と,``クォーテーション(Shift+7)''を2連で使う.

\subsection{環境}
さまざまな文書形式は「環境」で提供される.環境とは,\verb|\begin{環境名}|で始まり,\verb|\end{環境名}|で終わるものである.

\subsubsection{箇条書き}
文書中で事柄を列挙するときに使われる箇条書きの形式には,番号付き箇条書き,番号なし箇条書き,見出し付き箇条書きの3種類がある.

\newpage

\paragraph{番号付き箇条書き}
\begin{enumerate}
\item 項目1
\item 項目2
\end{enumerate}

\paragraph{番号なし箇条書き}
\begin{itemize}
\item 項目1
\item 項目2
\end{itemize}

\subsubsection{揃え}
文字揃えもよく使う機能である.左揃え,中央揃え,右揃えの3種類がある.
\begin{center}
\item 中央揃え
\end{center}
\begin{flushright}
\item 右揃え\\
複数行記述するときは,強制改行を使う
\end{flushright}

\subsubsection{その他の環境}
そのまま出力するverbatim環境は,プログラムをそのまま載せたいときなどに便利である.
\begin{verbatim*}
#include <stdio.h>
int main(void){
  printf(``Hello, world!\n'');
 return 0;
}
\end{verbatim*}
verbatim環境使用上の注意点としては,TAB文字は切り捨てられるため,字下げの空白がTAB文字で入力されている場合は,半角スペースで置き換える必要があることである.

\subsection{相互参照}
レポートで特に便利なのは,\verb|\label|コマンドと\verb|\ref|,\verb|\pageref|コマンドを用いた相互参照である.節や図,表,式の番号を本文宙で参照したいときに,参照した意識の直後にラベルを定義しておけば,本文中でそのラベルを参照するだけで,対応する番号を自動的に挿入することができる.

\textbf{相互参照の練習: }文字サイズの変更については,第\ref{sec:文字サイズ}節で練習した(1ページ).

なお相互参照を使っているときには,ラベルに変更があると,2回コンパイルをする必要がある.

\section{数式の練習}
本文中での式番号の参照も,相互参照の機能を使う.これにより,文章の順番を入れ替えた場合でも自動的に式番号が振りなおされ,正しい式番号が参照される.
\subsection{基本練習}
\subsubsection{本文中の数式}
関数$y = ax^{2} + bx + c$を描く.
\subsubsection{別行立て数式}
\paragraph{1行,式番号付き}
\begin{equation}
y = ax^{2} + bx + c
\end{equation}

\paragraph{複数行,式番号付き}
\begin{eqnarray}
y &=& ax^{2} + bx + c\\
x + y &<& -4\nonumber
\end{eqnarray}
式の最後に\verb|\nonumber|を書くことで,一部の式に番号を付けないようにもできる.

\subsection{数式の記述練習}
式(3)と式(4)は,式の変形の公式である.
\begin{eqnarray}
(a \pm b)^{2} &=& a^{2} \pm 2ab + b^{2}\\
(a \pm b)^{3} &=& a^{3} \pm 3a^{2}b + 3ab^{2} \pm b^{3}
\end{eqnarray}
式(5)は,三角関数の加法定理の公式である.
\begin{equation}
\sin(a \pm b) = \sin{\alpha}\cos{\beta} {\pm} \cos{\alpha}\sin{\beta}
\end{equation}

式(6)は,定積分の性質をを表している.
\begin{equation}
\int_{a}^{b} kf(x)\,dx = k\int_{a}^{b} f(x)\,dx (kは定数)
\end{equation}
次に,一般の2次方程式の解の公式を導出する.2次方程式
\begin{equation}
ax^{2} + bx + c = 0
\end{equation}
の両辺を$x^{2}$の係数$a$で割って,定数項を右辺に移行すれば,
\begin{eqnarray}
x^{2} + \frac{b}{a}x = -\frac{c}{a}\nonumber
\end{eqnarray}
となる.さて,左辺に完全平方式を作るために,両辺に$\displaystyle \left(\frac{b}{2a}\right)^{2}$を加えれば,
\begin{eqnarray}
x + a {\cdot}\frac{b}{2a}x + \left(\frac{b}{2a}\right)^{2} = \left(\frac{b}{2a}\right)^{2} -\frac{c}{a}\nonumber\\
\left(x + \frac{b}{2a}\right)^{2} = \frac{b^2 - 4ac}{4a^{2}}
\end{eqnarray}
となる.したがって,両辺の平方根を作れば,
\begin{eqnarray}
x + {\cdot}\frac{b}{2a} = {\pm}\frac{\sqrt{b^{2}-4ac}}{2a}\nonumber\\
x = \frac{-b {\pm}\sqrt{b^{2}-4ac}}{2a}
\end{eqnarray}
これが,式(7)の解を与える公式である.しかしながら,上の計算は,$D = b^{2} -4ac {\geq} 0$であるときに成り立つ.\\
しかし,複素数を用いれば,$D = b^{2} -4ac < 0$の場合にも,式(8)は解の公式となる.

\section{図と表の練習}
ここでは,これまでの実験レポートで扱った内容を題材に,図と表の作成と,{\TeX}文書への挿入について練習する.

図については,PowerPointを用いて図を作成し,Metafile to EPS Converter を用いて,EPSファイルに変換し,文書に挿入する方法を習得する.

表については,table環境およびteablar環境を用いて計測データの表を作成する手順を習得する.

さらに,Excelを用いてグラフを作成し,そのグラフをMetafile to EPS Converter を用いて,EPSファイルに変換し,文書に挿入する方法を習得する.

\subsection{実験内容}
図1は,オームの法則の実験回路である.この回路を用いて,一定電圧下で負荷抵抗の値を増加させたときの電流値を測定する.

\newpage

\subsection{実験結果}
前節の実験結果を表1に示す.

\begin{table}[h]
\caption{電流値の測定結果}
\begin{center}
\renewcommand{\arraystretch}{0.7}
\begin{tabular}{r||r|r||r|r}
\hline
印加電圧 &
\multicolumn{2}{c||}{5[V]} &
\multicolumn{2}{c}{10[V]}
\\\hline
負荷抵抗[kΩ] & 理論値[mA] & 実測値[mA] & 理論値[mA] & 実測値[mA]
\\\hline
0.1 & 50.0 & 49.9 & 100.0 & 99.1\\
0.2 & 25.0 & 24.1 & 50.0 & 49.9\\
0.3 & 16.7 & 16.5 & 33.3 & 32.5\\
0.4 & 12.5 & 12.5 & 25.0 & 24.3\\
0.5 & 10.0 & 10.0 & 20.0 & 20.0\\
0.6 & 8.3 & 8.4 & 16.7 & 16.2\\
0.7 & 7.1 & 7.1 & 14.3 & 14.3\\
0.8 & 6.3 & 6.2 & 12.5 & 12.5\\
0.9 & 5.6 & 5.6 & 11.1 & 11.1\\
1.0 & 5.0 & 5.0 & 10.0 & 9.9\\
\hline
\end{tabular}
\end{center}
\end{table}

表1をグラフに表すと図2のようになる.\\
グラフから,電圧が一定の条件下では,電流値は抵抗値に反比例していることがわかる.

\subsection{画像ファイルの挿入}
{\TeX}文書には,図3のように画像ファイルを挿入することも可能である.
\section{考察}
\subsection{WordのようなWYSIWYG形式のソフトウェアによる文書作成と,{\TeX}のようなマークアップ言語による文書作成の比較}
\subsection{実験テキストに盛り込むべき事項,説明が不十分と思われる個所について}

\section*{参考文献}
\begin{thebibliography}{99}
  \bibitem{竹部}竹部啓輔, 「{\TeX}によるレポート作成,令和3年度電子制御工学実験・3年前期テキスト pp.A-1-A-18, (2021/4.
\end{thebibliography}


\end{document}